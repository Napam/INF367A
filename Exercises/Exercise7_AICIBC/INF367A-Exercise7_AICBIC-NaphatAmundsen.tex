\documentclass[12pt]{article}
\usepackage[utf8]{inputenc}
\usepackage[left=2.5cm, right=2.5cm, top=2.0cm]{geometry}
\usepackage{sectsty}
\usepackage{graphicx}
\usepackage{amsmath}
\usepackage{amssymb}
\usepackage{undertilde}
\usepackage{kbordermatrix}
\usepackage{listings}
\usepackage{ulem}
\usepackage{soul}
% \usepackage{tikz}
% \usepackage{pgfplots}
% \pgfplotsset{compat=1.16} 
\usepackage{siunitx}
\usepackage{pythonhighlight}
\usepackage{caption}
\usepackage{float}
\usepackage{url}
\usepackage{enumitem}
\usepackage{bm}
\usepackage{empheq}
\usepackage{tcolorbox}
\usepackage{framed}
\usepackage{xparse}
\usepackage{algorithm, algorithmic}
% \usepackage{algorithmic}
\usepackage{booktabs}
\usepackage{tabularx}
\usepackage{hyperref}
\usepackage{mathtools}
% \usepackage[shortlabels]{enumitem}
\tcbuselibrary{breakable}


%  ---------------------- COMMANDS ---------------------------
% Double underline
\def\dunderline#1{\underline{\underline{#1}}}

% Shorten nonumber command
\def\nnb{\nonumber}

% Shorten boldsymbol command
\def\bs#1{\boldsymbol{#1}}

% Algorithmic newline
\def\algonewline{\STATE{ }}

% Eigenvectors of
\def\eigvecof{\text{eigenvectors of }}

% Eigenvalues of 
\def\eigvalof{\text{eigenvalues of }}

% Enclose in square brackets
\newcommand{\enclb}[1]{\left[#1\right]}
% Enclose in parenthesis
\newcommand{\enclp}[1]{\left(#1\right)}
% Enclose in curly brackets
\newcommand{\enclc}[1]{\left\{#1\right\}}

% Annotate first argument, text in second argument
\newcommand{\overtext}[2]{\overbrace{#1}^{\mathclap{\text{#2}}}}
\newcommand{\undertext}[2]{\underbrace{#1}_{\mathclap{\text{#2}}}}

% Normalization constant in multivariate normal
\newcommand{\mvnconst}[1]{\frac{1}{(2\pi)^{d/2} |#1|^{1/2}}}

% Exponential factor in multivariate normal
\newcommand{\mvnexpo}[3][x]{\exp \left[  -{\frac{1}{2}} (#1 - #2)^T #3^{-1} (#1 - #2) \right]}

% Multivariate normal
\newcommand{\mvn}[3][x]{\mvnconst{#3} \mvnexpo[#1]{#2}{#3}}

% For numbering in align* environment
\newcommand{\numberthis}{\addtocounter{equation}{1}\tag{\theequation}}

%  Sum notation w/ limits as argument and index as option
\newcommand{\sumlim}[3][i]{\sum\limits_{#1=#2}^{#3}}

%  Product notation w/ limits as argument and index as option
\newcommand{\prodlim}[3][i]{\prod\limits_{#1=#2}^{#3}}

%  Sum notation with only information beneath
\newcommand{\sumnolim}[1]{\sum\limits_{#1}}

%  Integral notation w/ limits as argument and index as option
\newcommand{\intlim}[2]{\int\limits_{#1}^{#2}}

%  Integral over whole domain notation, no args
\newcommand{\intinf}{\int\limits_{-\infty}^{\infty}}

%  Partial derivative notation, arg1: numerator, arg2: denominator
\newcommand{\pfrac}[3][ ]{\frac{\partial^{#1} #2}{\partial #3^{#1}}}

%  Derivative notation, arg1: numerator, arg2: denominator
\newcommand{\dvfrac}[3][ ]{\frac{\text{d}^{#1} #2}{\text{d} #3^{#1}}}

% When doing Gauss-Jordan, create arrow showing operations
\newcommand{\ro}[1]{\xrightarrow{\mathmakebox[\rowidth]{#1}}}

% ----------------------INVIRONMENTS---------------------------
% Item list with title
\newenvironment{titlemize}[1]{%
  \paragraph{#1}
  \begin{itemize}}
  {\end{itemize}}

  % Enum list with title
\newenvironment{titleenum}[1]{%
  \paragraph{#1}
  \begin{enumerate}}
  {\end{enumerate}}

  % Augmented matrix (matrix with vertical line)
\newenvironment{sysmatrix}[1]
  {\left(\begin{array}{@{}#1@{}}}
  {\end{array}\right)}

 % Make matrices with more spacing
\makeatletter
\renewcommand*\env@matrix[1][\arraystretch]{%
  \edef\arraystretch{#1}%
  \hskip -\arraycolsep
  \let\@ifnextchar\new@ifnextchar
  \array{*\c@MaxMatrixCols c}}
\makeatother

% \renewcommand*{\arraystretch}{1.5}

\newlength{\rowidth}% row operation width
\AtBeginDocument{\setlength{\rowidth}{3em}}

\floatname{algorithm}{Algorithm}
\renewcommand{\algorithmicrequire}{\textbf{Input:}}
\renewcommand{\algorithmicensure}{\textbf{Output:}}

\begin{document}
\title{\textbf{INF367A Exercise 7}}
\author{Naphat Amundsen}
\maketitle
\sectionfont{\fontsize{14}{15}\selectfont}
\subsectionfont{\fontsize{12}{15}\selectfont}
\subsubsectionfont{\fontsize{12}{15}\selectfont}
\graphicspath{ {./images/} }

\ifx
\begin{figure}[H]
	\centering
	\includegraphics[scale=0.8]{Figure_2}
	\caption{Insert caption here}
\end{figure}
\fi

\section*{Introduction}
This exercise is about doing Bayesian model selection. 

\section{Task}
\begin{tcolorbox}[breakable]
    We want to use linear models. But the data are not always linear and thus we may need to use basis functions to     handle non-linearity. However, it is not clear which basis functions are good for any particular data.  Furthermore, selection of the noise precision $\beta$ is crucial for to get good performance. 

    \vspace{5mm}
    In this task, we model (again) bicycling in New York. You should download the data set new\_york\_bicycles3.csv from MittUiB. The goal is to predict the highest daily temperature given the other variables. In this exercise, we select a set of basis functions and the value of $\beta$. To this end, you should try different basis functions and different values of $\beta$. In other words, a model is specified by the basis functions and beta,   i.e., $M_i = (\phi_i ,\beta_i ).$ Note: If the selected $\beta$ is either smallest or the largest value that you  tried, then you should probably try more values. Use the following model selection strategies: 

    \begin{enumerate}
        \item Use the posterior ratio to select a model. To do this, you need to compute marginal likelihoods.  (Note: you want to compare $P(D | M_i )P(M_i)$ values of different models.) 
        \item Use BIC to select a model. (Note: you want to maximize BIC.) 
        \item Use AIC to select a model. (Note: you want to minimize AIC.) 
        \item Use cross-validation to select a model. (Note: you want to minimize the validation loss.) 
    \end{enumerate}

    \textbf{Hints:} Typically, one projects data to higher dimensions. However, it may be useful to consider projecting data also to a lower dimensional space (i.e. not to use all features). 

    \vspace{3mm}
    Remember that when you compute marginal likelihood, you should use the predictive distribution $P(y | \hat{f_\theta}(x)) = N(y | m^T \phi(x),1/\beta + \phi(x)^T S\phi(x)).$ In case of AIC, BIC, and the cross-validation loss, you should use the the usual likelihood $P(y|w,x,\beta) = N(y|w^T \phi(x),1/\beta).$ 

    \vspace{3mm}
    Remember that the MAP estimate of a Gaussian distribution is its mean. 

    \vspace{3mm}
    Scipy's Gaussian distribution, norm has a scale parameter that takes in a standard deviation, i.e., p $\sqrt{1/precision}$. 

    \vspace{3mm}
    If you get totally insensible results, first sanity check is to check whether you are minimizing when you should be maximizing or vice versa.
\end{tcolorbox}
\section*{Methods:}
    Disclaimer: I am just writing what I think here in case I forget in the future, scroll down to find actual answers.
    \subsection*{Posterior ratio and BIC}
    Posterior ratio in this context is simply comparing the posteriors to all the models, i.e
    $P(D|M_i)P(M_i) / P(D)$, but is since $P(D)$ is a common factor for all, we can ignore it, thus needing only to compare $P(D|M_i)P(M_i)$. The models have parameters $\theta$ (in this case coefficients of the basis functions and the intercept), so we have to marginalize away $\theta$ in order to obtain $P(D|M_i)$: 
    \begin{align}
        P(D|M_i) = \int P(D|\theta, M_i)P(\theta|M_i)d\theta
    \end{align}
    when $n \rightarrow \infty$.

    \noindent The integral is not in a nice form, so it is approximated using Laplace approximation, which results in:
    \begin{align}
        \log P(D|M_i) \approx \sumlim{1}{n} \log P(D_i|\theta,M_i) + \log P(\theta|M_i) + \frac{d}{2}\log 2\pi - \frac{d}{2} \log n - \frac{1}{2} |F(\hat{\theta})| 
    \end{align}
    Then for large values of $n$ (which we already assume), several terms converge to constants, resulting with 
    \begin{align}
        \log P(D|\hat{\theta}, M) - \frac{d}{2}\log n \label{eq:bic}
    \end{align}
    where $\hat{\theta} = \arg\max_\theta P(D|\theta,M)P(\theta|M).$ 

    \noindent All of which we will define as $BIC(M) = \log P(D|\hat{\theta},M) - \frac{d}{2} \log n$

    \noindent In order to find $\hat{\theta}$ we can simply use the conjugate prior linear regression weights assuming that the prior weights are normally distributed as $\theta \sim \mathcal{N}(0,\alpha^{-1} I)$:
    \begin{align}
        S &= A^{-1} = \enclp{\alpha I + \beta \sumlim{1}{n} x_i x_i^T}^{-1} \\
        m &= A^{-1} b = \beta S \sumlim{1}{n} y_i x_i \\ 
        &\Rightarrow P(w|y,x,\alpha,\beta) = \mathcal{N}(w|m,S)
    \end{align}
    where $\beta$ is the error precision ($\epsilon_i \sim \mathcal{N}(0, \beta^{-1})$). Since the normal distribution peaks at its mean, $\hat{\theta}$ is simply the mean $m$. 

    \subsection*{AIC}
    Pure instrumentally, AIC is very similar to BIC (although the fundamental ideas are different). AIC is defined as 
    \begin{align}
        AIC = -\frac{2}{n} \log  P(y|\hat{\theta},X) + 2 \cdot \frac{d}{n}
    \end{align}
    Note that likelihood here is using a different notation than BIC. 

    \subsection*{Cross Validation}
    Just regular cross validation. The cross validation performed in this task is with 5 folds. The hyperparameter in interest is the degree of the polynomial basis function. 

\section{Actual answers}
I assume that $P(M_i) \sim \operatorname{Uniform}$. In the case of posterior ratio, the assumption effectively removes the prior ratio as they will always be $1$:
    \begin{align}
        &\frac{P(M_i|D)}{P(M_j|D)} = \frac{P(D|M_i)}{P(D|M_j)} \times \frac{P(M_i)}{P(M_j)} \\
        \Rightarrow &\frac{P(M_i|D)}{P(M_j|D)} = \frac{P(D|M_i)}{P(D|M_j)}
    \end{align}
Effectively, we are comparing which posteriors have the highest values, so I will just do that instead of finding the ratios between all combinations of things. Why make it difficult you know?

\def\b#1\textbf{#1}
The candidate models are polynomial models of degree 1 to 8. The results are as follows:
\begin{table}[H]
    \centering
    \caption{Model selection for polynomial fit. AIC and BIC uses $\hat{\theta}$ that was calculated the Bayesian way (conjugate prior). The Cross Validation calculates $\hat{\theta}$ using whatever Sklearn uses. The performance metric for CV is the negative mean absolute error (same error as AIC and BIC, but negative). Posterior: higher is better, BIC: higher is better, AIC: lower is better, CV higher is better. Best values with respect to method is bolded. \textbf{Note:} Since we assumed uniform prior for models, comparing posterior ratios is equivalent to comparing BIC values}
    \begin{tabular}{llrrrrr}
        \toprule
             &   &  LogLhoods &      BIC &   AIC &      CV\_val &  CV\_train \\
        beta & degree &            &          &       &             &           \\
        \midrule
        0.10 & 1 &   -1482.97 & -1499.07 & 13.92 &       -8.40 &     -6.89 \\
             & 2 &   -1143.85 & -1200.19 & 10.89 &       -8.75 &     -6.13 \\
             & 3 &    -967.78 & -1118.03 &  9.57 &      -16.13 &     -5.00 \\
             & 4 &    \b{-743.22}  & -1081.27 &  8.12 &     -148.72 &     -6.28 \\
             & 5 &   -5844.83 & -6520.95 & 56.98 &    -5583.39 &     -3.47 \\
             & 6 &   -5837.21 & -7076.75 & 58.87 &   -26028.28 &     -0.23 \\
             & 7 &   -5913.04 & -8037.97 & 62.66 &  -161911.32 &     -0.02 \\
             & 8 &   -5913.04 & -9366.04 & 67.29 & -1182049.76 &     -0.06 \\
        1.58 & 1 &   -3944.54 & -3960.63 & 36.92 &       -8.40 &     -6.89 \\
             & 2 &   -3893.18 & -3949.52 & 36.58 &       -8.75 &     -6.13 \\
             & 3 &   -3613.14 & -3763.39 & 34.29 &      -16.13 &     -5.00 \\
             & 4 &   -2405.22 & -2743.27 & 23.66 &     -148.72 &     -6.28 \\
             & 5 &   -5833.08 & -6509.19 & 56.87 &    -5583.39 &     -3.47 \\
             & 6 &   -5913.04 & -7152.58 & 59.58 &   -26028.28 &     -0.23 \\
             & 7 &   -5913.04 & -8037.97 & 62.66 &  -161911.32 &     -0.02 \\
             & 8 &   -5913.04 & -9366.04 & 67.29 & -1182049.76 &     -0.06 \\
        3.05 & 1 &   -4426.03 & -4442.13 & 41.42 &       -8.40 &     -6.89 \\
             & 2 &   -4399.65 & -4455.99 & 41.31 &       -8.75 &     -6.13 \\
             & 3 &   -4125.45 & -4275.70 & 39.08 &      -16.13 &     -5.00 \\
             & 4 &   -2901.54 & -3239.60 & 28.29 &     -148.72 &     -6.28 \\
             & 5 &   -5887.47 & -6563.59 & 57.38 &    -5583.39 &     -3.47 \\
             & 6 &   -5913.04 & -7152.58 & 59.58 &   -26028.28 &     -0.23 \\
             & 7 &   -5913.04 & -8037.97 & 62.66 &  -161911.32 &     -0.02 \\
             & 8 &   -5913.04 & -9366.04 & 67.29 & -1182049.76 &     -0.06 \\
        4.53 & 1 &   -4678.90 & -4694.99 & 43.78 &       -8.40 &     -6.89 \\
             & 2 &   -4657.49 & -4713.83 & 43.72 &       -8.75 &     -6.13 \\
             & 3 &   -4372.17 & -4522.42 & 41.38 &      -16.13 &     -5.00 \\
             & 4 &   -3207.27 & -3545.32 & 31.15 &     -148.72 &     -6.28 \\
             & 5 &   -5831.43 & -6507.54 & 56.85 &    -5583.39 &     -3.47 \\
             & 6 &   -5913.04 & -7152.58 & 59.58 &   -26028.28 &     -0.23 \\
             & 7 &   -5913.04 & -8037.97 & 62.66 &  -161911.32 &     -0.02 \\
             & 8 &   -5913.04 & -9366.04 & 67.29 & -1182049.76 &     -0.06 \\
        6.00 & 1 &   -4830.13 & -4846.23 & 45.20 &       -8.40 &     -6.89 \\
             & 2 &   -4837.83 & -4894.17 & 45.41 &       -8.75 &     -6.13 \\
             & 3 &   -4534.53 & -4684.78 & 42.90 &      -16.13 &     -5.00 \\
             & 4 &   -3407.26 & -3745.32 & 33.02 &     -148.72 &     -6.28 \\
             & 5 &   -5858.63 & -6534.75 & 57.11 &    -5583.39 &     -3.47 \\
             & 6 &   -5913.04 & -7152.58 & 59.58 &   -26028.28 &     -0.23 \\
             & 7 &   -5913.04 & -8037.97 & 62.66 &  -161911.32 &     -0.02 \\
             & 8 &   -5913.04 & -9366.04 & 67.29 & -1182049.76 &     -0.06 \\
        \bottomrule
    \end{tabular}
\end{table}
Posterior, BIC, AIC seem to agree that a polynomial kernel of degree 4 is the best. Cross validation says that we should have degree 1, and for some weird reason the mean training error is best at 7 degrees and not 8. 
% \bibliographystyle{apalike}
% \bibliographystyle{ieeetran}
% \bibliography{citations}
\appendix
\section*{Code}
\inputpython{models.py}{0}{1000}

\end{document}

